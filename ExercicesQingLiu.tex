	\documentclass[A4, 11pt]{article}
%packages
\usepackage{latexsym}
\usepackage{amsmath}
\usepackage{amsthm}
\usepackage{amssymb}
\usepackage{amsfonts}
\usepackage{esint}
\usepackage[all]{xy}
%\usepackage{dsfont}
\usepackage{graphics,graphicx}
\usepackage{accents}
\usepackage[french]{babel}
\usepackage[utf8]{inputenc}
\usepackage[T1]{fontenc}
\usepackage{color}

%definition, theoremes etc
\newtheorem{defin}{D\'efinition}[section]
\newtheorem{theo}[defin]{Th\'eor\`eme}
\newtheorem{prop}[defin]{Proposition}
\newtheorem{cor}[defin]{Corollaire}
\newtheorem{lem}[defin]{Lemme}
\newtheorem{rem}[defin]{Remarque}
\newtheorem{ex}[defin]{Exemple}
\newtheorem{prob}[defin]{Problème}
\newtheorem{exer}{Exercise}
\renewcommand{\theexer}{\empty{}} 
%ensembles de nombres 

\def\C{{\mathbb C}}
\def\E{{\mathbb E}}
\def\N{{\mathbb N}} 
\def\R{{\mathbb R}} 
\def\Q{{\mathbb Q}}
\def\Z{{\mathbb Z}}
\def\H{{\mathbb H}}
\def\K{{\mathbb K}}
\def\F{{\mathbb F}}
\def\U{{\mathbb U}}
\def\P{{\mathbb P}}
\def\Ker{  \operatorname{Ker} }
\def\Im{  \operatorname{Im} }
\def\Spec{ \operatorname{Spec}}
\author{Séverin Philip}
\begin{document}
\title{Exercises Qing Liu}
\maketitle

\section{General properties of Schemes}
\subsection{Reduced schemes and integral schemes}
\begin{exer} (4.2)
\end{exer}
\begin{proof}
Le morphisme canonique $\Spec \mathcal{O}_{X,x} \rightarrow X$ est donné par le morphisme 
$\mathcal{O}_{X}(U) \rightarrow \mathcal{O}_{X,x}$ pour un ouvert affine $U$ de $X$ contenant $x$. On note $\mathcal{O}_{X}(U)=A$ et l' morphisme est celui de localisation en $\mathfrak{p}$ idéal premier associé à $x$. Si $y$ est un point de $U$ qui se spécialise en $x$, $x\in \overline{\{y\}}$, par définition si $\mathfrak{q}$ est l'idéal premier associé à $y$, on a $\mathfrak{q} \subset \mathfrak{p}$ d'où $\mathfrak{q}$ est un idéal premier du localisé $A_{\mathfrak{p}}$. Par suite $y$ est dans l'image de $\Spec A_{\mathfrak{p}} \rightarrow \Spec A$. Il est clair que réciproquement un élément de cette image provient d'un idéal premier de $A_{\mathfrak{p}}$ et donc par localisation d'un idéal premier de $A$ inclus dans $\mathfrak{p}$ ce qui correspond à un point qui se spécialise en $x$. Comme le morphisme $\Spec \mathcal{O}_{X,x} \rightarrow X$ est indépendant du choix de $U$ (Pourquoi ?) cela suffit.

{\color{blue} A mon avis ça dépend pas du choix de l'ouvert car si tu prends un autre ouvert $V$, tu peux trouver un affine $W$ dans $U\cap V$. Alors tu écris un diagramme commutatif avec tous les $\mathcal{O}_X(X)$, $\mathcal{O}_X(U)$, $\mathcal{O}_X(V)$, $\mathcal{O}_X(W)$, les flèches de restrictions et les flèches vers $\mathcal{O}_{X,x}$. Ensuite tu appliques $\Spec$ et tu vois que tous les morphismes coïncident.}

\end{proof}

\begin{exer}(4.3)
\end{exer}
\begin{proof}
On a une inclusion $\mathcal{O}_K[T] \hookrightarrow K[T]$ qui induit un morphisme 
$j\colon \Spec K[T] \rightarrow \Spec \mathcal{O}_K [T]$. On montre que c'est une immersion ouverte. Si $\mathfrak{p}\in \Spec K[T]$, $j(\mathfrak{p})=\mathfrak{p}\cap \mathcal{O}_K[T]$. L'image de $j$ est $\Spec \mathcal{O}_K[T] \setminus V(t)$ qui est ouverte. En effet, si $t\in \mathfrak{p}\cap \mathcal{O}_K[T]$ alors $t\in \mathfrak{p}$ et $\mathfrak{p}=K[T]$ ce qui est impossible. Inversement, si $t\notin \mathfrak{p}$ avec $\mathfrak{p}$ idéal premier de $\mathcal{O}_K[T]$ alors par localisation en $S=\mathcal{O}_K[T]\setminus \{0\}$ \textcolor{blue}{($\mathcal{O}_K\setminus \{0\}$?)} on a $\mathfrak{p}K[T]$ idéal premier qui vérifie $\mathfrak{p}K[T]\cap \mathcal{O}_K[T]=\mathfrak{p}$. Il reste à voir que $j^{\sharp}_x$ est un isomorphisme en tout point $x\in \Spec K[T]$ ce qui est trivialement le cas (même une égalité).

{\color{blue} Yes je suis d'accord, en fait pour l'homéomorphisme on peut direct appliquer 2.1.7.c) avec $S=\mathcal{O}_K\setminus \{0\}$. Les morphismes entre les fibres sont bien des égalités je suis d'accord.}

 L'idéal $(T)$ est le seul point de $\Spec K[T]$ qui se spécialise en $(T,t)$. (Je crois ?)
 
{\color{blue} Je suis d'accord car cela revient à chercher les polynômes irréductibles $P$ de $K[T]$ tels que $(P)\cap \mathcal{O}_K[T]\subset (T, t)$. En localisant tu as nécessairement $T| P$ et donc $P=T$. Enfin je crois que c'est bon.}
\end{proof}
\begin{exer}(4.8)
\end{exer}
\begin{proof}
Soit $x$ un point de $X$ et $(U_i)$ les ouverts affines qui recouvrent $X$ (en nombre fini). On suppose que $x\in U_1$ quitte à renuméroté les ouverts. Le point $x$ correspond à un idéal premier $\mathfrak{p}$ contenu dans un idéal maximal $\mathfrak{m}$ de $\mathcal{O}_X(U_1)$ qui lui même correspond à un point fermé de $U_1$. On a donc l'existence de $x_1\in U_1$ fermé dans $U_1$ et $x_1\in \overline{\{x\}}$ la fermeture étant prise dans $X$. Si $x_1$ est fermé dans tous les autres $U_i$ qui le contiennent il est fermé dans $X$. Sinon il existe un $i\in \{2,\dots , n\}$ tel que $x_1\in U_i$ et $x_1$ n'est pas fermé dans $U_i$. On peut à nouveau supposer que $i=2$ et par le même argument qu'avant obtenir $x_2\in U_2$ fermé dans $U_2$ et $x_2\notin U_1$. En répétant le procédé au plus $n$ fois on obtient un point fermé dans tous les ouverts affines $U_j$ qui le contiennent.
\end{proof}
\begin{exer}(4.11)

\end{exer}
\begin{proof}
\begin{itemize}

\item[$(i)\Rightarrow (ii)$] On montre que $f^{\sharp}(U)$ est injectif pour tout ouvert affine $U$ de $Y$. Soit $g\in \mathcal{O}_Y(U)$ tel que $f^{\sharp}(U)(g)=0$. Pour tout $y=f(x)\in U\cap f(X)$ on a 
$$f^{\sharp}_x\colon \mathcal{O}_{Y,f(x)} \rightarrow \mathcal{O}_{X,x}$$
qui est un morphisme local et $f^{\sharp}_x(g)=0\in \mathfrak{m}_x$. D'où $g\in \mathfrak{m}_{f(x)}$. Or l'ensemble $\{y\in U, ~g\in \mathfrak{m}_y\}$ est un fermé de $U$, celui-ci contient $f(X)$ c'est donc $U$ tout entier. Il suit que $g\in \bigcap\limits_{\mathfrak{p}\in \Spec \mathcal{O}_X(U)} \mathfrak{p}$ est nilpotent. Comme $Y$ est réduit $g=0$. Le résultat est vrai sans l'hypothèse $U$ affine en prenant un recouvrement par des ouverts affine.
\item[$(ii)\Rightarrow (iii)$] Par la proposition 4.18 le morphisme $\mathcal{O}_X(U)\rightarrow \mathcal{O}_{X,x}$ est injectif pour tout $x\in U$ donc en particulier si $V\subset U$ est un ouvert, $\mathcal{O}_X(U)\rightarrow \mathcal{O}_X(V)$ est injectif. En effet le diagramme suivant commute
$$\xymatrix{
\mathcal{O}_X(U) \ar[r] \ar[dr] & \mathcal{O}_X(V) \ar[d] \\
 & \mathcal{O}_{X,x} 
}$$ 
Le résultat suit trivialement de cette remarque et de l'injectivité de $\mathcal{O}_Y(V) \rightarrow \mathcal{O}_X(f^{-1}(V))$ par $(ii)$.
\end{itemize}

\item[$(iii)\Rightarrow (iv)$] Soit $V$ un ouvert de $Y$ contenant $f(\xi_X )$. Le diagramme suivant commute et par $(iii)$ les flèches sont injectives.
$$\xymatrix{
\mathcal{O}_Y(V) \ar[r] \ar[d] & \mathcal{O}_X(f^{-1}(V)) \ar[d]\\
\mathcal{O}_{Y,f(\xi_X)} \ar[r] & \mathcal{O}_{X,\xi_X} 
}$$
Comme $\xi_X$ est le point générique de $X$ qui est un schéma entier (integral ?) son idéal maximal associé est $(0)$. Par injectivité et le fait que $f^{\sharp}_{\xi_X}$ est local l'idéal maximal de $f(\xi_X)$ est donc lui même $(0)$. Il suit que $f(\xi_X)=\xi_Y$.

\item[$(iv)\Rightarrow (v)$] Trivial.
\item[$(v)\Rightarrow (i)$] Comme $Y$ est un schéma entier $\overline{\{\xi_Y\}}=Y$.
 
\end{proof}

\section{Morphisms and base change}

\subsection{The technique of base change}

\begin{prop} 1.4 Démonstration du point d.
\end{prop}
 \begin{proof}
 On considère $U,V$ des sous-schémas ouvert de $X$ et $Y$. Il faut vérifier que $i\times j$ induit un isomorphisme de $U\times_S V$ dans $p^{-1}(U)\cap q^{-1}(V)$. Soit $Z$ un schéma et $f,g$ des morphismes $Z\rightarrow U$, $Z\rightarrow V$. En composant avec les injections de $U,V$ dans $X$ et $Y$ on obtient un diagramme commutatif
 $$\xymatrix{
Z \ar[dr] \ar[ddr]_{i\circ f} \ar[rrd]^{j\circ g} & & \\
& X\times_S Y \ar[r] \ar[d] & Y \ar[d]\\
 & X \ar[r] & S 
 }$$
 Il suit que la flèche du milieu se factorise par $p^{-1}(U)\cap q^{-1}(V)$. Comme le morphisme $i\times  j$ est l'unique morphisme de $U\times_S V$ dans $X\times_S Y$ faisant commuter les diagrammes et se factorisant par $p^{-1}(U)\cap q^{-1}(V)$ c'est un isomorphisme $U\times_S V\simeq p^{-1}(U)\cap q^{-1}(V)$.
 \end{proof}
 \begin{exer}(1.7)
 \end{exer}
 \begin{proof}
 On suppose $X,Y$ et $S$ affines, c'est-à-dire $X=\Spec M$, $Y=\Spec N$ et $S=\Spec R$. Le résultat dans le cas général suit du cas affine par recollement (Intuitivement ok, l'idée doit marcher mais un truc détaillé serait bien...).
On note $f\colon R\rightarrow M$, $g\colon R\rightarrow N$.
 Soit $(\mathfrak{p}, \mathfrak{q})\in X\times Y$ tels que $\mathfrak{p}\in X_s$, $\mathfrak{q}\in Y_s$ pour un point $s\in S$. On a donc $f^{-1}(\mathfrak{p})=s$ d'où les morphismes 
 $$\xymatrix{
R \ar[r]& M \ar[r] & M/\mathfrak{p} \\
 }$$ 
 induisent
  $$\xymatrix{
R/s \ar[r]\rightarrow & M/\mathfrak{p} \\
k(s) \ar[r] & k(\mathfrak{p})
 }$$ 
 et il en est de même pour $\mathfrak{q}$ et $N$. On a donc des morphismes $M\rightarrow k(\mathfrak{p})$ et $N\rightarrow k(\mathfrak{q})$ tel que le diagramme suivant commute
 $$\xymatrix{
k(\mathfrak{p}) \otimes_{k(s)} k(\mathfrak{q}) & & \\
 & M \otimes_R N \ar@{.>}[ul] & N \ar[l] \ar@/_1pc/[llu] \\
 &  M \ar[u] \ar@/^1pc/[luu] & R \ar[l] \ar[u]  
 }$$
 et donc par propriété du produit tensoriel on obtient l'existence de la flèche en pointillé d'où un morphisme naturel
 $$\Spec \big( k(\mathfrak{p}) \otimes_{k(s)} k(\mathfrak{q}) \big) \rightarrow X\times_S Y.$$
 On vérifie maintenant que l'image de ce morphisme est contenu dans l'ensemble 
 $$\{z\in X\times_S Y, ~p(z)=\mathfrak{p}, q(z)=\mathfrak{q}\}.$$
 Il faut vérifier que si $I$ est un idéal premier de $k(\mathfrak{p}) \otimes_{k(s)} k(\mathfrak{q})$ alors $\varphi^{-1}(I)$ est l'idéal $\mathfrak{p}$ de $M$ où $\varphi$ est l'application $M\rightarrow k(\mathfrak{p}) \otimes_{k(s)} k(\mathfrak{q})$. Comme $\varphi(\mathfrak{p})=0$ on a une inclusion. Maintenant, si $m\in M\setminus \mathfrak{p}$ est tel que $\varphi(m)\in I$ alors comme $\varphi(m)=\overline{m}\otimes 1$ qui est inversible dans $k(\mathfrak{p}) \otimes_{k(s)} k(\mathfrak{q})$ ce qui est impossible car alors $I=k(\mathfrak{p}) \otimes_{k(s)} k(\mathfrak{q})$. Donc $\varphi^{-1}(I)=\mathfrak{p}$ et ce raisonnement appliqué à $N$ et $\mathfrak{q}$ assure l'inclusion. 
 
 Il faut maintenant voir qu'un idéal $I$ de $M\otimes_R N$ tel que $i^{-1}(I)=\mathfrak{p}$ et $j^{-1}(I)=\mathfrak{q}$ où $i,j$ sont les applications $M\rightarrow M\otimes_R N$, $N\rightarrow M\otimes_R N$ est tel que $M\otimes_R N \rightarrow k(I)$ se factorise par $k(\mathfrak{p}) \otimes_{k(s)} k(\mathfrak{q})$. En effet, $\mathfrak{p}\otimes 1$ est donc dans $I$ et est envoyé sur $0$ dans $k(I)$ donc on a une factorisation
 $$M\otimes_R N\rightarrow M/\mathfrak{p} \otimes_R N/ \mathfrak{q} \rightarrow k(I).$$
 Il reste à voir que l'on peut étendre cette dernière flèche à $k(\mathfrak{p}) \otimes_{k(s)} k(\mathfrak{q})$. Il suit donc une factorisation de $k(I)\rightarrow X\times_S Y$ en
 $$ \xymatrix{
 k(I)\ar[r] & \Spec \big(k(\mathfrak{p}) \otimes_{k(s)} k(\mathfrak{q})\big)\ar[r] & X\times_S Y. 
 }$$
 \end{proof}
 
 \begin{exer}(1.8)
 \end{exer}
 \begin{proof}
 C'est une conséquence de l'exercice précédent. Soit $y\in Y$, il existe un $s\in S$ tel que $y\in Y_s$. Par surjectivité de $X\rightarrow S$ la fibre $X_s$ au dessus de $s$ est non vide donc contient un point $x\in X$. Par l'exercice 1.7 l'ensemble 
$$\{z\in X\times_S Y, ~p(z)=\mathfrak{p}, q(z)=\mathfrak{q}\}$$
est homéomorphe à  $\Spec \big( k(\mathfrak{p}) \otimes_{k(s)} k(\mathfrak{q}) \big)$ qui est non vide donc contient au moins un point. Le morphisme $q\colon X\times_S Y$ est donc surjectif.
  \end{proof}
 \begin{exer}(1.10)
 \end{exer}
 \begin{proof}
 Par la propriété universelle du produit fibré en tant qu'ensembles les applications $p\colon X\times_S Y \rightarrow X$ et $q\colon X\times_S Y \rightarrow Y$ donnent l'existence d'une unique application continue $f\colon |X\times_S Y| \rightarrow |X|\times_{|S|} |Y|$. Cette application est surjective par l'exercice 1.7.
 
 On considère le produit tensoriel $\C\otimes_{\R} \C$. On a 
 $$\C\otimes_{\R}\C= \C\otimes_{\R} \R[X]/(X^2+1)= \C[X]/(X^2+1)=\C(X)/(X+i)(X-i)$$
 et ce dernier anneau est isomorphe à $\C\times \C$ (Spécifier l'isomorphisme). 
 Comme il n'y a qu'un point dans $\Spec \C$ le produit fibré des deux ensembles $|\Spec \C|$ sur $|\Spec \R|$ ne contient qu'un seul point. Par contre $\Spec \big(\C \times \C \big)$ contient deux idéaux premiers $(1,0)$ et $(0,1)$. L'application $f$ est donc surjective mais pas injective ces deux points du produit fibré de schémas ayant même image dans le produit fibré d'ensembles. 
 \end{proof}
 \subsection{Applications to algebraic varieties}
 \begin{exer} (2.4)
 \end{exer}
 \begin{proof}
 On va considérer le cas où $S=\Spec k$ pour un corps $k$. Comme $Y$ est de type finie sur $k$, pour $U$ un ouvert affine de $Y$, $\mathcal{O}_Y(U)$ est une $k$-algèbre de type finie. 
 
 Pour un ouvert affine $V$ de $X$ contenant $x$ on a un morphisme canonique $\Spec \mathcal{O}_{X,x} \rightarrow V$ provenant d'un morphisme $\mathcal{O}_X(V) \rightarrow \mathcal{O}_{X,x}$. 
 On a $f_x^{\sharp}(U)\colon \mathcal{O}_Y(U) \rightarrow \mathcal{O}_{\Spec \mathcal{O}_{X,x}}(f_x^{-1}(U))$. Or $\mathcal{O}_{\Spec \mathcal{O}_{X,x}}(f_x^{-1}(U))$ correspond à une localisation de $\mathcal{O}_{X,x}$ et comme $\mathcal{O}_Y(U)$ est une $k$-algèbre de type finie, l'image de $f_x$ qui est un $k$-morphisme est déterminé par l'image des générateurs de $\mathcal{O}_Y(U)$ sur $k$. Soient $y_1,\dots, y_n$ ces générateurs et $\frac{f_i}{g_i}$ leurs images. Soit $g$ le produit des $g_i$, $D(g)$ est un ouvert affine principal $W$ contenant $x$ de $V$ et l'on a $\frac{f_i}{g_i}\in \mathcal{O}_X(W)$. On peut donc définir le morphisme $f_U$ de $U$ dans $V$ tel que $f_U\circ i_x=f_x$. On peut définir des morphismes $f_U$ de cette façon pour tout ouvert affine $U$ de $Y$ qui se recolle par construction et obtenir le morphisme $f$ souhaité.
 \end{proof}
 \subsection{Some global properties of morphisms}
 \begin{exer}(3.1)
 \end{exer}
 \begin{proof}
 Par hypothèse les morphismes $f_i\colon f^{-1}(Y_i)\rightarrow Y_i$ sont des immersions fermés et se recollent. Comme $f(X)$ est fermé il suit que $f$ est une immersion fermée topologique. Il reste à voir que les applications locales sur les faisceaux sont surjectives. Or c'est un problème local et on peut donc se restreindre à $Y_i$ où le résultat vient à nouveau de l'hypothèse sur les $f_i$. 
 \end{proof}
 \begin{exer}(3.2)
 \end{exer}
 \begin{proof}
 \begin{itemize}
 \item[$(iii)\Rightarrow (ii)$] Tout morphisme de $X$ dans un schéma $Y$ est séparé, c'est en particulier le cas du morphisme vers $\Spec \Z$ qui est un schéma affine.
  \item[$(ii)\Rightarrow (i)$] La composition de morphismes séparés est séparé et tout morphisme entre schéma affines est séparé. Par hypothèse il existe $f\colon X\rightarrow \Spec A$ séparé et on a $\Spec A \rightarrow \Spec \Z$ séparé, donc $X\rightarrow \Spec \Z$ est séparé.
   \item[$(i)\Rightarrow (iii)$] Je n'ai pas réussi mais je pense qu'il faut voir qu'il y a un lien entre le produit fibré sur $\Z$ et sur un schéma $Y$ et obtenir la diagonale de l'un comme image réciproque de la diagonale de l'autre. 
 \end{itemize}
 \end{proof}
 \begin{exer}(3.10)
 \end{exer}
 \begin{proof}
 On a un diagramme commutatif
$$\xymatrix{
X \ar@/_/[ddr]_{Id_X} \ar[rd]^{(Id_X,f)} \ar@/^2pc/[rrd]^{f}  & & \\
& X\times_S Y \ar[r]^{q} \ar[d]_{p} & Y\ar[d] \\
& X\ar[r] & S \\
}$$
Le triangle de gauche est commutatif donc $p\circ (Id_X,f)=Id_X$.


On a un autre diagramme commutatif
$$\xymatrix{
X\times_S Y \ar@/_/[ddr]_{f\circ p} \ar[rd]^{\varphi} \ar@/^2pc/[rrd]^{q}  & & \\
& Y\times_S Y \ar[r]^{q_Y} \ar[d]_{p_Y} & Y\ar[d] \\
& Y\ar[r] & S \\
}$$

On vérifie que $\Gamma_f$ l'image de $(Id_X,f)$ est $\varphi^{-1}(\Delta_Y)$. 
Si $x\in X$, l'élément $(Id_X,f)(x)$ est déterminé uniquement (Est-ce vrai ???) par ces deux égalités 
$$p( (Id_X,f)(x))= x;$$
$$q ( (Id_X,f)(x))=f(x).$$
Or si $x\in \varphi^{-1}(\Delta_Y)$, on a $p_Y(\varphi(x))=q_Y(\varphi(x))$ car $\varphi(x)\in \Delta_Y$. D'où
$$f\circ p (x)=p_Y (\varphi (x))=q_Y(\varphi(x))=q(x).$$
Donc $x=(Id_X,f)(p(x))$ par la caractérisation précédente. La réciproque est claire en remontant les égalités. 
 \end{proof}
 \section{Some local properties}
 \subsection{Normal schemes}
 \begin{exer}(1.4)
 \end{exer}
 \begin{proof}
 Si $X$ est normal alors il est normal en tout point donc en particulier pour les points fermés.
 
 Soit $x\in X$ un point qui n'est pas fermé. Alors par l'exercice 2.4.8 il existe un point fermé $y$ dans $\overline{\{x\}}$. Soit $V$ un ouvert affine contenant $y$, alors si $x\notin V$ on aurait $x\in X\setminus V$ qui est fermé donc en particulier $\overline{\{x\}}\subset X\setminus V$ et donc $y \in X\setminus V$ ce qui est une contradiction. Il suit que $x\in V$ et que l'on obtient $\mathcal{O}_{X,x}$ par localisation de $\mathcal{O}_{X,y}$. Ce dernier est donc réduit, intègre ou normal si $\mathcal{O}_{X,y}$ l'est ce qui prouve l'implication.
 \end{proof}
 
 \begin{exer}(1.9)
 \end{exer}
 \begin{proof}
 On considère $A$ un anneau de Dedekind et $X=\Spec A$. Soit $x_0\in \Spec A$ un point fermé et $U=X\setminus \{x_0\}$ un ouvert. On note $\mathfrak{m}_0$ l'idéal associé à $x_0$ et $t$ un générateur de $\mathfrak{m_0}$ dans $A_{\mathfrak{m}_0}$. L'idéal $(t)$ se décompose en produit d'idéaux premiers car $A$ est un anneau de Dedekind donc 
 $$(t)= \mathfrak{m}_0 \prod\limits_{i=1}^{n} \mathfrak{m}_i^{a_i}$$
 et $V(t)=\{x_0,\dots x_n\}$ les points $x_i$ étant ceux des idéaux $\mathfrak{m}_i$.
 On note $t_i$ un générateur de $\mathfrak{m}_i$ dans $A_{\mathfrak{m_i}}$. On peut choisir $t_i$ tel que $t_i\notin \mathfrak{m}_0$. En effet, si $\mathfrak{m}_i\setminus \mathfrak{m}_i^2 \subset \mathfrak{m}_0$ on aurait $\mathfrak{m}_i\subset \mathfrak{m}_0$ et donc égalité par maximalité. On a $t=ut_i^{a_i}$ dans $A_{\mathfrak{m_i}}$ où $u$ est inversible donc $t\cdot t_i^{-a_i}=u$ et par suite $f=t^{-1}t_i^{a_i}\prod\limits_{i\neq j} t_j^{a_j} =u'\in A_{\mathfrak{m}_i}=\mathcal{O}_{X,x_i}$. Il existe donc des ouverts $U_i$ contenant $x_i$ tels que $f\in \mathcal{O}_X(U_i)$  et $U_i$ ne contient pas $x_0$. Il est de plus clair que $f\in \mathcal{O}_X(X\setminus V(t))$. Il suit que $f\in \mathcal{O}_X(U)$ car les $U_i$ et $X\setminus V(t)$ forment un recouvrement ouvert de $U$.
 \end{proof}
 
\end{document}